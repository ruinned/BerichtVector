\section*{Kurzfassung}
Im Rahmen dieser Bachelorarbeit wird zun�chst eine Literaturrecherche �ber bestehende Analysemethoden durchgef�hrt, mit deren Hilfe die L�sung einer multikriteriellen Optimierungsaufgabe hinsichtlich der Drehmoment- und Ger�uschbildung am Dieselmotor realisiert werden kann.

Aufbauend wird ein ausgew�hltes Verfahren benutzt, um ein Optimum des Motordrehmoments bei m�glichst niedrigen Ger�uschemissionen zu finden. Die Optimierung wird anhand eines am Institut f�r Automatisierungstechnik entwickelten Simulationsmodells eines Dieselmotors durchgef�hrt und am institutseigenen Motorenpr�fstand validiert. 

\textbf{Schl�sselw�rter:} Dieselmotor, Drehmomentbildung, Ger�uschemissionen, Steuerung, multikriterielle Optimierung, Wirkungsgrad.


\selectlanguage{english}
\section*{Abstract}

As part of this bachelor thesis a literature research about existing analytical methods is initially performed. The latter enables one to realize the solution of a multi-criterial optimization task in regards to torque and noise generation.

On the basis of this a selected procedure is used to identify the optimum of the engine torque at lowest noise generation. The optimization is performed for a diesel engine with the help of a simulation model developed by the Institute of Automatic Control an Mechatronics at the TU Darmstadt ( Institut f�r Automatisierungstechnik) and is being validated on a engine test bed provided by the institute.


\textbf{Keywords:} Dieselmotor, motortorque, motornoise, optimization, control, multi-criterial optimization, motor effectiveness.
\selectlanguage{ngerman} 
\section*{Aufgabenstellung}
%
Die T�tigkeit und die w�hrend des gesamten Industriepraktikums bearbeiteten Aufgaben bestanden vorwiegend darin, den Umgang mit unterschiedlichen Softwaretools zu erlernen, die w�hrend der Entwicklung von verschiedenen Software-Produkten in der Abteilung eingesetzt werden.

Folgende Themen wurden w�hrend der gesamten T�tigkeit behandelt:
%
\begin{enumerate}
\item \textbf{MISRA- und QA-C-Analyse zum Umstieg auf neue Versionen}
\item \textbf{BTE auf Hardware}
\item \textbf{FlexRay Transceiver Test Suite}
\item \textbf{PDuR Test Generator ASR3 Removal}
\item \textbf{AWK-Programmierung und XSLT-Transformation}
\end{enumerate}
%
\vspace{0.5cm}
\vspace{0.5cm}
\vspace{0.5cm}
\begin{tabular}{ll}
Beginn: & \SADABegin \\
Ende:   & \SADAAbgabe \\
\end{tabular}

\vspace{1cm}

\begin{tabular}{ll}
\rule{7cm}{0.4pt} \hspace{1cm} & \rule{7cm}{0.4pt} \\
\SADABetreuer & \SADABetreuerII
\end{tabular}

\vfill
{\renewcommand{\baselinestretch}{1} % f�r diesen Abschnitt einfacher Zeilenabstand
\normalsize % anwenden des Zeilenabstandes
\begin{minipage}{0.6\textwidth}
	\SADAUnternehmen\\[0.5cm]
	Ingersheimer Str. 24\\
	70499 Stuttgart\\	
	\SADAwebsiteVector
\end{minipage}
\begin{minipage}{0.2\textwidth}
\flushright  % rechtsb�ndig
\ \\[2.7cm]
\SADAlogoVector\;
\end{minipage}}


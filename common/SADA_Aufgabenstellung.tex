\section*{Aufgabenstellung}
%
Moderne Dieselmotoren werden bei der Applikation hinsichtlich verschiedenster Kriterien optimiert. Neben den Schadstoff-Emissionen z�hlen hierzu sowohl das Motordrehmoment, der Wirkungsgrad als auch die Ger�uschemissionen.

Im Rahmen dieser Bachelorarbeit soll zun�chst auf Basis einer Literaturrecherche untersucht werden, wie aus dem Zylinderdruckverlauf Kenngr��en generiert werden k�nnen, die es erm�glichen, das Drehmoment, den Wirkungsgrad als auch die Ger�uschemissionen am Beispiel des Dieselmotors zu quantifizieren. Diese gewonnenen Gr��en sollen hierauf aufbauend modellbasiert und rechnergest�tzt mono- als auch multikriteriell lokal optimiert werden. F�r ausgesuchte Stellgr��en soll zus�tzlich analysiert werden, in welchem Ma�e sie das Pareto-Problem entsch�rfen k�nnen. Als Basis f�r diese Optimierung muss zun�chst ein Gesamtmodell des Motors erstellt werden, dessen beide Teilmodelle, das Luftpfad- und Verbrennungsmodell, bereits zur Verf�gung stehen.
%



\vspace{0.5cm}
\begin{tabular}{ll}
Beginn: & \SADABegin \\
Ende:   & \SADAAbgabe \\
Seminar:& \SADASeminar
\end{tabular}

\vspace{1cm}

\begin{tabular}{ll}
\rule{7cm}{0.4pt} \hspace{1cm} & \rule{7cm}{0.4pt} \\
\SADAProf & \SADABetreuer\\
 &\SADABetreuerII
\end{tabular}

\vfill
{\renewcommand{\baselinestretch}{1} % f�r diesen Abschnitt einfacher Zeilenabstand
\normalsize % anwenden des Zeilenabstandes
\begin{minipage}{0.6\textwidth}
	Technische Universit�t Darmstadt\\
	\SADAinstitut\\[0.5cm]
%
	Landgraf-Georg-Stra�e 4\\
	64283 Darmstadt\\
	Telefon \SADAtel\\
	\SADAwebsite
\end{minipage}
\begin{minipage}{0.2\textwidth}
\flushright  % rechtsb�ndig
\ \\[2.7cm]
\SADAlogo\;
\end{minipage}}


\selectlanguage{ngerman}
\maketitle

% Die Farbe der Identit�tsleiste wird auf Grau umgestellt, damit nicht alle Seiten
% farbig gedruckt werden m�ssen
\ifTUDdesign
	\ifOnlyColorFront	% ggf. nachfolgende Balken andere Farbe zuweisen
		\makeatletter 	% ben�tigt, um die @-Befehle auszuf�hren
    \renewcommand{\@TUD@largerulecolor}{\color{tud9c}}% am besten gleiche Farbe wie in der ersten Zeile und die Zahl durch die 0 ersetzen, dann hat das Grau die richtige Intensit�t
		\makeatother		
	\fi	
\fi


\pagenumbering{roman}	% Bis zum Hauptteil werden r�mische Seitenzahlen verwendet

% =================================================================================
% Spezielle Seiten f�r studentische Arbeiten
% =================================================================================
\cleardoublepage
\section*{Aufgabenstellung}
%
Die T�tigkeit und die w�hrend des gesamten Industriepraktikums bearbeiteten Aufgaben bestanden vorwiegend darin, den Umgang mit unterschiedlichen Softwaretools zu erlernen, die w�hrend der Entwicklung von verschiedenen Software-Produkten in der Abteilung eingesetzt werden.

Folgende Themen wurden w�hrend der gesamten T�tigkeit behandelt:
%
\begin{enumerate}
\item \textbf{MISRA- und QA-C-Analyse zum Umstieg auf neue Versionen}
\item \textbf{BTE auf Hardware}
\item \textbf{FlexRay Transceiver Test Suite}
\item \textbf{PDuR Test Generator ASR3 Removal}
\item \textbf{AWK-Programmierung und XSLT-Transformation}
\end{enumerate}
%
\vspace{0.5cm}
\vspace{0.5cm}
\vspace{0.5cm}
\begin{tabular}{ll}
Beginn: & \SADABegin \\
Ende:   & \SADAAbgabe \\
\end{tabular}

\vspace{1cm}

\begin{tabular}{ll}
\rule{7cm}{0.4pt} \hspace{1cm} & \rule{7cm}{0.4pt} \\
\SADABetreuer & \SADABetreuerII
\end{tabular}

\vfill
{\renewcommand{\baselinestretch}{1} % f�r diesen Abschnitt einfacher Zeilenabstand
\normalsize % anwenden des Zeilenabstandes
\begin{minipage}{0.6\textwidth}
	\SADAUnternehmen\\[0.5cm]
	Ingersheimer Str. 24\\
	70499 Stuttgart\\	
	\SADAwebsiteVector
\end{minipage}
\begin{minipage}{0.2\textwidth}
\flushright  % rechtsb�ndig
\ \\[2.7cm]
\SADAlogoVector\;
\end{minipage}}



\cleardoublepage
\ \\[3cm]	% Diese Zeile erzeugt einen Abstand von 4cm zur ersten Zeile, die nur ein Leerzeichen
			% enth�lt

\ifx\SADAVarianteErklaerung\ETIT
	\section*{Erkl�rung}
	\noindent
	Hiermit versichere ich, dass ich die vorliegende Arbeit ohne Hilfe Dritter und nur mit den angegebenen Quellen und Hilfsmitteln angefertigt habe. Alle Stellen, die aus den Quellen entnommen wurden, sind als solche kenntlich gemacht. Diese Arbeit hat in gleicher oder �hnlicher Form noch keiner Pr�fungsbeh�rde vorgelegen.\vspace*{20mm} \\
	\noindent
	\begin{tabular}{ll}
		\SADAStadt, den \SADAAbgabe	\hspace{1cm}	& \rule{0.4\textwidth}{0.4pt}\\
										            & \SADAAutor
	\end{tabular}
	


\else\ifx\SADAVarianteErklaerung\MBDA
	\section*{Erkl�rungen}
	\noindent
	Hiermit erkl�re ich an Eides statt, dass ich die vorliegende \SADATyp\ mit dem Titel\ \glqq\SADATitel\grqq\ selb\-st�ndig und ohne fremde Hilfe verfasst, andere als die angegebenen Quellen und Hilfsmittel nicht benutzt und die aus anderen	Quellen entnommenen Stellen als solche gekennzeichnet habe.\\
	Diese Arbeit hat in gleicher oder �hnlicher Form noch keiner Pr�fungsbeh�rde vorgelegen.\vspace*{20mm} \\
	\noindent
	\begin{tabular}{ll}
		\SADAStadt, den \SADAAbgabe	\hspace{1cm}	& \rule{0.4\textwidth}{0.4pt}\\
										& \SADAAutor
	\end{tabular}
	
	
	\vspace{40mm}
	\noindent
	Ich bin damit einverstanden, dass die TU Darmstadt das Urheberrecht an meiner \SADATyp\ zu wissenschaftlichen Zwecken nutzen kann.\vspace*{20mm} \\
	\noindent
	\begin{tabular}{ll}
		\SADAStadt, den \SADAAbgabe	\hspace{1cm}	& \rule{0.4\textwidth}{0.4pt}\\
										& \SADAAutor
	\end{tabular}

	{\huge Hier fehlt noch was!}
	
	
\else\ifx\SADAVarianteErklaerung\MBSA
	\section*{Erkl�rungen}
	\noindent
	Hiermit erkl�re ich an Eides statt, dass ich die vorliegende \SADATyp\ mit dem Titel\ \glqq\SADATitel\grqq\ selb\-st�ndig und ohne fremde Hilfe verfasst, andere als die angegebenen Quellen und Hilfsmittel nicht benutzt und die aus anderen	Quellen entnommenen Stellen als solche gekennzeichnet habe.\\
	Diese Arbeit hat in gleicher oder �hnlicher Form noch keiner Pr�fungsbeh�rde vorgelegen.\vspace*{20mm} \\
	\noindent
	\begin{tabular}{ll}
		\SADAStadt, den \SADAAbgabe	\hspace{1cm}	& \rule{0.4\textwidth}{0.4pt}\\
										& \SADAAutor
	\end{tabular}
	
	
	\vspace{40mm}
	\noindent
	Ich bin damit einverstanden, dass die TU Darmstadt das Urheberrecht an meiner \SADATyp\ zu wissenschaftlichen Zwecken nutzen kann.\vspace*{20mm} \\
	\noindent
	\begin{tabular}{ll}
		\SADAStadt, den \SADAAbgabe	\hspace{1cm}	& \rule{0.4\textwidth}{0.4pt}\\
										& \SADAAutor
	\end{tabular}


\else
	{\huge Unbekannte Variante der Erkl�rung!}

\fi\fi\fi






\clearpage
\section*{Kurzfassung}
%
Der vorliegende Bericht beschreibt lediglich ein Teil des Industriepraktikums bei der Vector Informatik GmbH, welches zwischen dem 01.04.2016 und dem 31.10.2016 stattfand. Die Abteilung PES1 erm�glichte die Durchf�hrung der T�tigkeit.

Es werden lediglich folgende Aufgaben hierbei dokumentiert:
%
\begin{enumerate}
\item \textbf{MISRA- und QA-C-Analyse zum Umstieg auf neue Versionen}
\item \textbf{BTE auf Hardware}
\end{enumerate}

\textbf{Schl�sselw�rter:} Vector Informatik GmbH, PES, MISRA, Basic Test Environment, eingebettete Systeme, Hardwarenahe Softwareentwicklung.


\selectlanguage{english}
\section*{Abstract}

This report describes a part of the internship, which took place at the company Vector Informatik GmbH between april and september 2016. The PES1 department made further this internship possible.

Only the following two tasks are documented in this report:
%
\begin{enumerate}
\item \textbf{MISRA- and QA-C analysis to estimate a possible upgrade to the actual standards.}
\item \textbf{Porting BTE on Hardware.}
\end{enumerate}

\textbf{Keywords:} Vector Informatik GmbH, PES, MISRA, Basic Test Environment, embedded systems, embedded software development.
\selectlanguage{ngerman} 
% =================================================================================

% =================================================================================
% Inhaltsverzeichnis
% =================================================================================
\cleardoublepage	% Auf einer leeren rechten Seite beginnen
\phantomsection		% Diese und die n�chste Zeile dient dazu, f�r das Inhalts-
					% verzeichnis einen Eintrag in das pdf-Inhaltsverzeichnis,
					% aber nicht in das normale Verzeichnis zu erzeugen.
\pdfbookmark[0]{\contentsname}{pdf:toc}	
\tableofcontents	% Inhaltsverzeichnis einf�gen
\clearpage	% Sonst kommt nichts mehr auf die Seite
% =================================================================================


% =================================================================================
% Symbole und Abk�rzungen
% =================================================================================
% Nach dem Inhaltsverzeichnis kommt ein Verzeichnis der h�ufig verwendeten
% Symbole und Abk�rzungen. Dazu kann man das Paket 'nomencl' verwenden, oder man
% erstellt es von Hand.
\chapter*{Symbole und Abk�rzungen}
\addcontentsline{toc}{chapter}{Symbole und Abk�rzungen} % erzeugt einen Eintrag im Inhaltsverzeichnis
%
\paragraph*{Abk�rzungen}
\begin{tabularx}{\textwidth}{@{}l@{\qquad}X}
K�rzel & vollst�ndige Bezeichnung \\ \midrule
  BTE & Basis Test Enviroment	 \\
	MISRA & Einlassbeh�lter / Einspritzbeginn\\
	GUI & engl. Graphical User Interface\\
	IDE & engl. Integrated Development Environment\\
	EPIC & engl. Eclipse Perl Integration\\
	BSW & Basissoftware \\
	AUTOSAR & ???? \\
	CUT & engl. component under test (Einspritzung) \\
	CDT & C/C++ Development Tooling \\
	VTR & ????\\
	ECU & ???\\
	ES & Echtzeitsystem (\textit{embedded system})\\
	HIS & Herstellerinitiative \\	
	MISRA & The Motor Industry Software Reliability Association\\
	PERL & Practical Extraction and Reporting Lenguage \\
	XML & Extensible Markup Language \\
	UML & Unified Modeling Language\\
	PES & Development Embedded Software \\
	Stub & Platzhalter  \\
	OEM & Original Equipment Manufacturer\\
	Z & Zylinder
\end{tabularx}
%
\cleardoublepage

% =================================================================================
% Hauptteil
% =================================================================================
\pagenumbering{arabic}	% Hauptteil bekommt arabische Seitenzahlen
\chapter{Einf�hrung}\label{cha:Intro}
%
\section{Firmenprofil}\label{cha:firmenprofi}
%
Die Vector Informatik GmbH unterst�tzt Hersteller und Zulieferer der Automobilindustrie und verwandter Branchen mit einer professionellen und offenen Plattform aus Werkzeugen, Softwarekomponenten und Dienstleistungen zur Entwicklung von eingebetteten Systemen.

Unter anderem geh�rt die PES-Abteilung zu dem Unternehmen. Diese stellt Erstausr�stern (OEMs) und Zulieferern der Automobilindustrie und verwandter Branchen Komponenten und Dienste bereit, um eingebettete Systeme zu erstellen. Sie liefert somit fundamentale Softwaretools zur Erg�nzung der eigenen Applikationssoftware der Kunden. 
%
\section{Hauptthemen der T�tigkeit}\label{cha:firmenprofi}
%
Das Ziel meiner T�tigkleit bei Vector bestand vorwiegend darin, den Umgang mit unterschiedlichen Softwaretools zu erlernen, die w�hrend der Entwicklung von verschiedenen Software-Produkten in der Abteilung eingesetzt werden.

Folgende Themen wurden mit der umfassenden Unterst�tzung von meinem Betreuer, Herrn Markus Schwarz, behandelt:

\begin{description}
\item \textbf{MISRA-Standard}: \\ Untersuchung eines m�glichen Umstiegs auf MISRA 2012 mit QA-C-9.
\item \textbf{BTE auf Hardware}:\\ Das existierende Testframework BTE wird so angepasst und erweitert, dass dieses direkt auf einer Embedded Plattform lauff�hig ist.
\item \textbf{FlexRay Transceiver Test Suite}:\\ FlexRay-Transceiver-Treiber lassen sich �ber das BTE-Testframework stub-basiert testen, ohne dass die entsprechende Hardware vorhanden ist. Im Rahmen dieser Aufgabe soll eine geeignete Methode gew�hlt werden, um eine geignete Kommunikation zwischen dem BTE-Tool, welches auf dem PC laufen soll, und dem Embedded-Prozessor aufzubauen und die Testhardware zu steuern.
\end{description}

\clearpage
\chapter{Numerische Optimierungsverfahren}\label{cha:NumerOptim}
%
Die in den Kapiteln ~\charef{cha:BeschrModell} beschriebenen Simulationsmodelle haben eine nicht lineare Struktur. Durch die Zusammenschaltung beider entsteht ein komplexes, nicht lineares Gesamtmodell das als Black-Box-Modell gehandhabt werden muss. In der Verwendung mit MATLAB\&Simulink kamen f�r die Optimierungsaufgabe zwei bereits im MATLAB-Code implementierte Funktionen in Frage. Zum einen ein lokales Optimierungsverfahren namens OPTIMIZE und zum anderen ein globales Verfahren mit der Bezeichnung GODLIKE. In den n�chsten beiden Unterkapiteln soll grundlegendes Hintergrundwissen �ber die beiden Optimierungsverfahren vermittelt werden.
%
\section{OPTIMIZE}
%
Die Funktion OPTIMIZE ist eine Erweiterung der Funktion FMINCON. Sie basiert auf dem Downhill-Simplex-Verfahren nach John Nelder und Roger Mead. Dieses Verfahren geh�rt zur Gruppe der Hill-Climb-Suchverfahren und kommt wie oben gefordert ohne die Berechnung von Ableitungen aus. Bei diesem Verfahren wird im N-dimensionalen Suchraum ein Konstrukt aus N+1 Punkten gebildet. Bei einem 2-dimensionalen Optimierungsproblem w�re das ein Dreieck, im 3-dimensionalen ein Tetraeder. Dieses Konstrukt, genannt Simplex, bekommt durch einen Algorithmus eine Verhaltensregel auferlegt die es zu einem lokalen Minimum streben l�sst. 
%
\section{GODLIKE}
%
Der GODLIKE\footnote{GODLIKE steht f�r Global Optimum Determination by Linking and Interchanging Kindred Evaluators.}-Algorithmus verwendet methaheuristische\footnote{Methaheuristische Verfahren stehen f�r Algorithmen zur n�herungsweisen L�sung eines kombinatorischen Optimierungsproblems und definieren abstrakte Folgen von Schritten, die auf beliebige Problemstellungen anwendbar sind.} Optimierungsverfahren. Er berechnet zun�chst eine Summe an zuf�lligen Funktionswerten, die als Polpulation bezeichnet werden. Aus diesen Startwerten bzw. Individuen dieser Population werden �ber ausgesuchte Wahrscheinlichkeitsverfahren neue Individuen gebildet, die eine neue Population bilden. Die Werte der Individuen konvergieren hierbei mit hoher Wahrscheinlichkeit gegen ein Minimum. 
\newpage
GODLIKE verwendet f�r die Suche eines Minimums vier Algorithmen, die im Folgenden erw�hnt werden sollen.

\begin{enumerate}
  \item GA(Genetic Algorithm)
	\item DE(Differential Evolution)
	\item ASA(Adapted Simulated Annealing)
	\item PSO(Particle Swarm Optimization)
\end{enumerate}
%
\subsection{Multikriterielle Optimierung}
%

\section{Wahl des Optimierungsalgorithmus}\label{sec:Wahl}
%
Der GODLIKE-Algorithmus wird zur Suche eines globalen Minimums verwendet und bringt demnach den Vorteil, dass ein m�glichst optimaler Stellgr��envektor gefunden wird, dessen Einsatz im zu optimierenden Modell ein m�glichst gro�es Drehmoment hervorrufen kann, wobei auch die Nebenbedingung einer geringen Ger�uschemission gut eingehalten wird. Dagegen spricht die Tatsache, dass dieser Algorithmus nicht wie der lokale Alogorithmus einen Weg zum Minimum, sondern die vollst�ndige Fl�che des vorgegebenen Suchraums absucht, was dazu f�hrt, dass er erheblich mehr Rechenresourcen ben�tigt. Aus diesem Grund kann der GODLIKE-Optimierungsalgorithmus nur f�r Offline-Simulationen verwendet werden, was im Rahmen dieser Arbeit kein gro�es Problem darstellt.

Der Downhill-Simplex-Algorithmus ist ein lokal suchendes Optimierungsverfahren. Der Simplex wandert daher im L�sungsraum von seinem vorgegebenen Startwert aus zu dem n�chst gelegenen Minimum. Grundlegender Nachteil hierbei ist, dass das lokale Minimum in der Regel nicht dem globalen Minimum entspricht. Ein klarer Vorteil ist jedoch die geringe Anzahl an Funktionsaufrufen die f�r die L�sungsfindung notwendig ist. Ein entscheidender Einflussfaktor bei der Verwendung dieser Funktion ist die Wahl der Startwerte. Zum einen ist die Anzahl der notwendigen Funktionsaufrufe zum Auffinden des lokalen Minimums um so kleiner je n�her die Startwerte am Minimum plaziert sind. Zum anderen kann durch die Wahl der Startwerte der Simplex so im Suchraum positioniert werden, dass er mit hoher Wahrscheinlichkeit in das globale Minimum l�uft und nicht in einem lokalen Minimum h�ngen bleibt. Da man in der Regel am globalen Minimum
interessiert ist, ist es bei der Verwendung dieses Algorithmus n�tig, ausreichendes Hintergrundwissen �ber das Optimierungsproblem zu besitzen, um entsprechende Anfangswerte w�hlen zu k�nnen.

Der Einfachheithalber wird im Rahmen dieser Arbeit der globale Optimierungsalgorithmus GODLIKE verwendet, dessen Ergebnisse mit Hilfe des Downhill-Simplex-Algorithmus best�tigt werden k�nnen, indem man ihm zur Suche des lokalen bzw. globalen Minimums entsprechende Anfangswerte vorgibt.

F�r n�here Informationen �ber die GODLIKE- bzw. OPTIMIZE-Optimierungsalgorithmen wird auf ~\cite{Zydek} bzw. ~\cite{GODLIKE} verwiesen.

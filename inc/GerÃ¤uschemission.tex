\chapter{Ger�uschentstehung bei Verbrennungsmotoren}\label{cha:Ger�uschemission}
%
Zur Einf�hrung in die Thematik der Schallentstehung bei den Verbrennungsmotoren insbesondere beim Dieselmotor und der Erkl�rung der Zusammenh�nge in der Motorakustik werden zun�chst einige grundlegende Begriffe der Schallentstehung nach ~\cite{Finger} vorgestellt.

Man kann die Schallentstehung in direkte und indirekte Schallentstehung untergliedern.
\begin{description}
\item[ \textnormal{Bei der} direkten Schallenstehung] werden die Teilchen der die Maschine umgebenden Luft durch Str�mungsvorg�nge in der Maschine zu Schwingungen, d.h. zu Luftschall, angeregt. Hierzu k�nnen die Str�mungsger�usche von L�ftern oder Luftpulsationen in Ansaugsystemen als Beispile angegeben werden.%\\
\item[\textnormal{Unter} der indirekten Schallentstehung] versteht man den Mechanismus, bei dem ausgehend von den Erregerkraften die Struktur eine K�rperschallanregung erf�hrt. Ein Teil der K�rperschall-Schwingungsenergie wird an der Strukturoberfl�che auf das angrenzende Umgebunsmedium �bertragen.
\end{description}
%
In der Motorakustik besteht beim Dieselmotor die M�glichkeit, durch gezielte Einstellung der Einspritzparameter die Verbrennung und somit den Zylinderdruckverlauf als anregende Kraft zu beeinflussen. Diese l��t sich jedoch nicht beliebig variieren, da allein durch den vorgegebenen
Betriebspunkt (Drehzahl und Last) eine Grundanregung gegeben ist. Durch moderne Einspritzsysteme l��t sich jedoch eine Vielzahl von Parametern ver�ndern, die eine direkte Auswirkung auf die Verbrennung und somit auf die Kraftanregung der Motorstruktur haben. Ver�nderungen des Verbrennungsablaufes haben jedoch auch Auswirkungen auf die gebildeten Schadstoffemissionen und den spezifischen Kraftstoffverbrauch und k�nnen daher nur in bestimmten Grenzen durchgef�hrt werden. Weiteres Potential zur Reduzierung des Motorger�usches liegt in der Beeinflussung des K�rperschall�bertragungsverhaltens der Motorstruktur, was aber nicht dem Ziel dieser Arbeit entspricht. 

Hier muss ich noch die Entstehung bzw. den Zusammenhang beschreiben, den es zwischen Ger�uschentstehung und Druckgradienten gibt...

Ger�uschemission Beidl 14-10

Um den Einfluss des Druckverlaufs auf die Ger�uschemission zu untersuchen, wurde im Rahmen dieser Arbeit eine Embendded MATLAB Function erstellt, die aus dem Druckverlauf $p_Z(\phi)$ den Druckgradienten $\frac{dp_Z}{d\phi}$ berechnet. Diese ist im Anhang... zu finden. 
Die zeitliche �nderung des Druckverlaufs ist nach den in \ref{cha:Ger�uschemission} gewonnenen Erkenntnissen ma�gebend f�r die Entwicklung von Ger�usch am Dieselverbrennungsmotor.

In Vorarbeiten wie ~\cite{Finger} wurde festgestellt, dass nicht nur der Einsatz einer Voreinspritzung der Ger�uschbildung entgegenwirkt, sondern auch die Kraftstoffmenge, die bei der Voreinspritzung in den Brennraum eingebracht wird. Somit 
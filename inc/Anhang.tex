\chapter{ Auschnitt der Softwaredokumentation}\label{cha:tools}
%
Im Rahmen dieser Arbeit sind folgende Software-Tools entwickelt worden:
\begin{itemize}
\item \textbf{Schwerpunktlage der Verbrennung} \\
\verb+SPL = function_SPL(phi_mod,QB,abtastzeit_s,nMot)+ berechnet.
\item \textbf{Zylinderdruckgradient} \\
 \verb+[dp_dphi,ddp_ddphi]  = Grad_dGrad_pZMax(phi,pZ)+
\end{itemize}
%
Die oben genannten Tools werden im Folgenden n�her erl�utert.
%
\section{Datei�bertragung }
%

\begin{lstlisting}[style=PERL_st, caption = {PERL Datei�bertragung},label={lst:PERL_Dateiueb}]

#!/Dwimperl/perl/bin/perl.exe -w
use strict;
use File::Copy;
use Cwd;

my $workingdir =cwd();	# find the directory where the script is being run
my $target_loc = "$workingdir/Relevant";	# set the target directory for the files
mkdir $target_loc;

# file where the paths of the source and header files are indicated 
open (PRJ_QAC , "$workingdir/Folder.prj")
or die "Fehler beim Oeffnen von '$workingdir/Baseline/Folder.prj': $!\n";

my @zeilen = <PRJ_QAC>;
my @var_arr;
my $var_str;
# hilfsvariablen
my $h1 = 0;	# wird der Anfangpunkt der Source-Pfade
my $h2 = 0;	
my $h3 = 0;	# wird der Endpunkt der Source-Pfade

# es muss ermoeglicht werden, den Punkt zu finden, ab dem die Source-Pfade ausgelesen werden koennen

foreach(@zeilen){	# iteriere ueber die Einstellungsdatei
	
	my $hola = 4;
	
	if($_ =~ /CompPers/){	# suche nach dem patern
		$h3 = $h1;	# wurde der Patern gefunden, dann halte die Zeilenummer -h1- fest und fange eine neue Suche ab dem Ort an
		for (my $p = $h1+1 ;$p < @zeilen ; $p++){
			if($zeilen[$p] =~ /EndContainedFilesMarker/){
				last;	# verlasse innere for-Schleife
			}
		$h3++
		}
		last;	# verlasse aeussere for-Schleife
	}	
	$h1++;
}

for(my $t = $h1+1;$t <= $h3; $t ++ )	{ #wird ueber Datei iteriert
	
	
	@var_arr = split(/\\/,$zeilen[$t]);	# trenne Zeile in Array ueber \ Zeichen
	$var_str = $var_arr[$#var_arr];		# bekomme letzter Eintrag des Arrays
	
	
	$h2 = $zeilen[$t];
	$h2 =~ tr/\\/\//;
		
	$target_loc = $target_loc."/$var_str";
	
	chomp($target_loc);
	chomp($h2);

	print "$target_loc\n";	# gebe aus, welche Dateien kopiert werden


	copy($h2, $target_loc) or die "File cannot be copied.";
	
	$target_loc = "$workingdir/Relevant";
	
}
\end{lstlisting}
%
\newpage
\section{Bearbeitung einer XML-Datei}
%
F�r die Echtzeitbestimmung der Kenngr��en Druckgradientenverlauf $\frac{dp}{d\phi}$ und deren zeitlichen Ableitung $\frac{d^2p}{d\phi^2}$ sind die momentanen Werte des simulierten Zylinderdrucks $p_Z$ und des Kurbelwinkels $\phi$ n�tig. Diese Kenngr��en werden vom Verbrennungsmodell ermittelt und k�nnen dementsprechend verarbeitet werden. In Abbildung \ref{fig:DruckgradSimulink} ist die entsprechende Embendded MATLAB Function zu erkennen. Der jeweilige Quellcode wird im Listing \ref{lst:Druckgrad} aufgezeigt.
%

\begin{lstlisting}[style=PERL_st, caption = {PERL XML},label={lst:PERL_XML}]
#!/Dwimperl/perl/bin/perl.exe -w
use strict;
use Cwd;

my $working = cwd();
my $string1 = "  \<file target=\"C\" name\=\"";
my $string2 = "\" folder\=\"=$working/Relevant\"\/\>";

my $z1 = 0;

# file where the paths of the source and header files are indicated 
open(LESEN1,"$working/Baseline/Folder.prj")
or die "Fehler beim Oeffnen von 'Folder.prj': $!\n";
# default file where the default information of the qac9 project is indicated
open(LESEN2,"prqaproject.xml")
or die "Fehler beim Oeffnen von 'prqaproject.xml': $!\n";

my @zeilen_files = <LESEN1>;
my @zeilen_xml = <LESEN2>;

close(LESEN1);
close(LESEN2);
unlink "prqaproject.xml";	# delete default file, afterwards a new file will be again created


my @var_arr;		#store the splited strings
my @var_arr_files;	#store the strings of the source files
my $var_str;
my $hilfsvar = 0;
my $h1 = 0;	# wird der Anfangpunkt der Source-Pfade
my $h2 = 0;	
my $h3 = 0;	# wird der Endpunkt der Source-Pfade
my $h1_xml = 0;
my $h2_xml = 0;


# es muss ermoeglicht werden, den Punkt zu finden, ab dem die Source-Pfade ausgelesen werden koennen
foreach(@zeilen_files){	# iteriere ueber die Einstellungsdatei
	
	if($_ =~ /CompPers/){	# suche nach dem Muster
		$h3 = $h1;	# wurde der Muster gefunden, dann halte die Zeilenummer -h1- fest und fange eine neue Suche ab dem Ort an
		for (my $p = $h1+1 ;$p < @zeilen_files ; $p++){
			if($zeilen_files[$p] =~ /EndContainedFilesMarker/){
				last;	# verlasse innere for-Schleife
			}
		$h3++
		}
		last;	# verlasse aeussere for-Schleife
	}	
	$h1++;
}


for(my $t = $h1+1;$t <= $h3; $t ++ )	{ #wird ueber Datei iteriert
	
	@var_arr = split(/\\/,$zeilen_files[$t]);	# trenne Zeile in Array ueber \ Zeichen
	# print $zeilen_files[$t]."\n";
	$var_str = $var_arr[$#var_arr];		# bekomme letzter Eintrag des Arrays, also den Namen der c-Datei
	
	$var_arr_files[$t-$h1-1] = $var_str;	# speichere die Namen der Datein in den entsprechenden Eintrag	
	
}	


foreach(@zeilen_xml){	# iteriere ueber die Einstellungsdatei
	
	if($_ =~ /\<\!-- Explicit files... --\>/){	# suche nach dem Muster
		$h2_xml = $h1_xml;	# wurde der Muster gefunden, dann halte die Zeilenummer -h1- fest und fange eine neue Suche ab dem Ort an
		for (my $p = $h1_xml+1 ;$p < @zeilen_xml ; $p++){
			if($zeilen_xml[$p] =~ /\<\/prqaproject\>/){
				last;	# verlasse innere for-Schleife
			}
		$h2_xml++
		}
		last;	# verlasse aeussere for-Schleife
	}	
	$h1_xml++;
}	

print $h1_xml."\n";
print $h2_xml."\n";
	
open(SCHREIBEN,"> prqaproject.xml")
or die "Fehler beim Oeffnen von 'prqaproject.xml': $!\n";


foreach(@zeilen_xml){
	
	if($z1 < $h1_xml+1){	# schreibe die Anfangszeilen zu der Datei
		print SCHREIBEN $zeilen_xml[$z1];
	} else {
		foreach(@var_arr_files){	# schreibe die Namen der Files in die Zieldatei
			chomp($_);
			print SCHREIBEN "$string1".$_."$string2\n";
		}
		print SCHREIBEN " \<\/files\>\n\<\/prqaproject\>";		
		last;
	}
	$z1++;	
}

close(SCHREIBEN) or die "Fehler beim Schliessen von 'prqaproject.xml': $! \n";
\end{lstlisting}
%
\newpage
\section{Erstellung eines XML-Testreports aus Bin�rdatei}
%
\begin{lstlisting}[style=PERL_st, caption = {Auschnitt vom File BteBin2Vtr.pl},label={lst:PERL_XML_Binary}]
#!/perl/bin/perl.exe -w

use strict;
use warnings;
use Time::Piece;

my @binFile_content;
my $binFile_length = 0;
my $binFile_index = 0;
my %event_map;
my $testcaseIsOpen = 0;

ConvertBin2Vtr();

#--------------------------------------------------------------
# Main function
sub ConvertBin2Vtr
{
  my $binFile_path = $ARGV[0];
  my $vtrFile_path = $ARGV[1];
  
  # read the transformation rules from the configuration file
  LoadEventConfig($ARGV[2]);

  # get the relevant data
  GetBinaryContent($binFile_path);

  # process the binary data and write to vtr
  OpenVtr($vtrFile_path);
  ProcessBinaryData();
  CloseVtr();
}

#--------------------------------------------------------------
# read the config file and store its content in event map
sub LoadEventConfig($)
{
  my $configFile_path = $_[0];
  
  # open config file
  open(configFile,$configFile_path) or die "Error while opening config file\n";  
  my @configFile_content = <configFile>;
  close(configFile);

  # store content of each line in event map
  # line layout: id(4 chars) text(x chars)
  # Example: 0105 This is my text
  foreach my $cfgFile_line (@configFile_content)
  {
    chomp($cfgFile_line);
    if (length($cfgFile_line) > 4)
    {
      my $event_id = substr($cfgFile_line,0,4);
      my $event_text = "";
      $event_text = substr($cfgFile_line,5);
      $event_map{$event_id} = $event_text; 
    }
  }
}

#--------------------------------------------------------------
sub GetBinaryContent($)
{
  my $binFile_path = $_[0];

  # open binary file for reading
  $binFile_length = -s "$binFile_path";
  open(binFile,"$binFile_path") or die "Error: opening binary file: $binFile_path\n";
  binmode(binFile);

  # read content
  my $binFile_string;
  read(binFile, $binFile_string, $binFile_length);
  close(binFile);
  
  # short the content (until EOF marker)
  @binFile_content = split("", $binFile_string);
  $binFile_length = GetIndexOfEOF($binFile_string);
  if ($binFile_length == -1) 
  {
    print "Warning: EOF-symbol not found\n";
  }
  else
  {
    @binFile_content = @binFile_content[0..$binFile_length];
  }
}

#--------------------------------------------------------------
sub ProcessBinaryData($)
{
  # iterate over binary data
  # data layout: kind(1 byte) data(x byte)
  do
  {
    my $data_kind = (BinFile_Read(1))[0];

    # detect what kind of data is present. Process data accordingly.
    if( $data_kind eq chr(hex 'FF'))
    {
      # TCASE
      # layout: kind (1 byte) id(2 byte)
      Process_TCASE(BinFile_Read(2));
    }
    elsif($data_kind eq chr(hex('FD')) || $data_kind eq chr(hex('FE')))
    {
      # message
      # type is coded in "kind"
      # layout: kind(1 byte) time(4 byte)
      my $message_kind=GetMessageType($data_kind);
      Process_Message($message_kind, BinFile_Read(4));
    }
    else
    {
      # event
      # type and size are coded in "kind"
      # layout: kind(1 byte) id(2 byte) time(4 byte) data(4*dlc byte)
      my $event_type = GetEventType($data_kind);
      my $event_size = GetEventSize($data_kind);
      my @event_data = BinFile_Read(6 + 4*$event_size);
      Process_Event($event_type, @event_data);
    }
  } while($binFile_index < $binFile_length);
}
...
\end{lstlisting}
\chapter{Softwaredokumentation}\label{cha:tools}
%
Im Rahmen dieser Arbeit sind folgende Software-Tools entwickelt worden:
\begin{itemize}
\item \textbf{Schwerpunktlage der Verbrennung} \\
\verb+SPL = function_SPL(phi_mod,QB,abtastzeit_s,nMot)+ berechnet.
\item \textbf{Zylinderdruckgradient} \\
 \verb+[dp_dphi,ddp_ddphi]  = Grad_dGrad_pZMax(phi,pZ)+
\end{itemize}
%
Die oben genannten Tools werden im Folgenden n�her erl�utert.
%
\section{Datei�bertragung }
%

\begin{lstlisting}[style=PERL_st, caption = {PERL Datei�bertragung},label={lst:PERL_Dateiueb}]

#!/Dwimperl/perl/bin/perl.exe -w
use strict;
use File::Copy;
use Cwd;

my $workingdir =cwd();	# find the directory where the script is being run
my $target_loc = "$workingdir/Relevant";	# set the target directory for the files
mkdir $target_loc;

# file where the paths of the source and header files are indicated 
open (PRJ_QAC , "$workingdir/Folder.prj")
or die "Fehler beim Oeffnen von '$workingdir/Baseline/Folder.prj': $!\n";

my @zeilen = <PRJ_QAC>;
my @var_arr;
my $var_str;
# hilfsvariablen
my $h1 = 0;	# wird der Anfangpunkt der Source-Pfade
my $h2 = 0;	
my $h3 = 0;	# wird der Endpunkt der Source-Pfade

# es muss ermoeglicht werden, den Punkt zu finden, ab dem die Source-Pfade ausgelesen werden koennen

foreach(@zeilen){	# iteriere ueber die Einstellungsdatei
	
	my $hola = 4;
	
	if($_ =~ /CompPers/){	# suche nach dem patern
		$h3 = $h1;	# wurde der Patern gefunden, dann halte die Zeilenummer -h1- fest und fange eine neue Suche ab dem Ort an
		for (my $p = $h1+1 ;$p < @zeilen ; $p++){
			if($zeilen[$p] =~ /EndContainedFilesMarker/){
				last;	# verlasse innere for-Schleife
			}
		$h3++
		}
		last;	# verlasse aeussere for-Schleife
	}	
	$h1++;
}

for(my $t = $h1+1;$t <= $h3; $t ++ )	{ #wird ueber Datei iteriert
	
	
	@var_arr = split(/\\/,$zeilen[$t]);	# trenne Zeile in Array ueber \ Zeichen
	$var_str = $var_arr[$#var_arr];		# bekomme letzter Eintrag des Arrays
	
	
	$h2 = $zeilen[$t];
	$h2 =~ tr/\\/\//;
		
	$target_loc = $target_loc."/$var_str";
	
	chomp($target_loc);
	chomp($h2);

	print "$target_loc\n";	# gebe aus, welche Dateien kopiert werden


	copy($h2, $target_loc) or die "File cannot be copied.";
	
	$target_loc = "$workingdir/Relevant";
	
}
\end{lstlisting}
%
\newpage
\section{Bearbeitung einer XML-Datei}
%
F�r die Echtzeitbestimmung der Kenngr��en Druckgradientenverlauf $\frac{dp}{d\phi}$ und deren zeitlichen Ableitung $\frac{d^2p}{d\phi^2}$ sind die momentanen Werte des simulierten Zylinderdrucks $p_Z$ und des Kurbelwinkels $\phi$ n�tig. Diese Kenngr��en werden vom Verbrennungsmodell ermittelt und k�nnen dementsprechend verarbeitet werden. In Abbildung \ref{fig:DruckgradSimulink} ist die entsprechende Embendded MATLAB Function zu erkennen. Der jeweilige Quellcode wird im Listing \ref{lst:Druckgrad} aufgezeigt.
%

\begin{lstlisting}[style=PERL_st, caption = {PERL XML},label={lst:PERL_XML}]
#!/Dwimperl/perl/bin/perl.exe -w
use strict;
use Cwd;

my $working = cwd();
my $string1 = "  \<file target=\"C\" name\=\"";
my $string2 = "\" folder\=\"=$working/Relevant\"\/\>";

my $z1 = 0;

# file where the paths of the source and header files are indicated 
open(LESEN1,"$working/Baseline/Folder.prj")
or die "Fehler beim Oeffnen von 'Folder.prj': $!\n";
# default file where the default information of the qac9 project is indicated
open(LESEN2,"prqaproject.xml")
or die "Fehler beim Oeffnen von 'prqaproject.xml': $!\n";

my @zeilen_files = <LESEN1>;
my @zeilen_xml = <LESEN2>;

close(LESEN1);
close(LESEN2);
unlink "prqaproject.xml";	# delete default file, afterwards a new file will be again created


my @var_arr;		#store the splited strings
my @var_arr_files;	#store the strings of the source files
my $var_str;
my $hilfsvar = 0;
my $h1 = 0;	# wird der Anfangpunkt der Source-Pfade
my $h2 = 0;	
my $h3 = 0;	# wird der Endpunkt der Source-Pfade
my $h1_xml = 0;
my $h2_xml = 0;


# es muss ermoeglicht werden, den Punkt zu finden, ab dem die Source-Pfade ausgelesen werden koennen
foreach(@zeilen_files){	# iteriere ueber die Einstellungsdatei
	
	if($_ =~ /CompPers/){	# suche nach dem Muster
		$h3 = $h1;	# wurde der Muster gefunden, dann halte die Zeilenummer -h1- fest und fange eine neue Suche ab dem Ort an
		for (my $p = $h1+1 ;$p < @zeilen_files ; $p++){
			if($zeilen_files[$p] =~ /EndContainedFilesMarker/){
				last;	# verlasse innere for-Schleife
			}
		$h3++
		}
		last;	# verlasse aeussere for-Schleife
	}	
	$h1++;
}


for(my $t = $h1+1;$t <= $h3; $t ++ )	{ #wird ueber Datei iteriert
	
	@var_arr = split(/\\/,$zeilen_files[$t]);	# trenne Zeile in Array ueber \ Zeichen
	# print $zeilen_files[$t]."\n";
	$var_str = $var_arr[$#var_arr];		# bekomme letzter Eintrag des Arrays, also den Namen der c-Datei
	
	$var_arr_files[$t-$h1-1] = $var_str;	# speichere die Namen der Datein in den entsprechenden Eintrag	
	
}	


foreach(@zeilen_xml){	# iteriere ueber die Einstellungsdatei
	
	if($_ =~ /\<\!-- Explicit files... --\>/){	# suche nach dem Muster
		$h2_xml = $h1_xml;	# wurde der Muster gefunden, dann halte die Zeilenummer -h1- fest und fange eine neue Suche ab dem Ort an
		for (my $p = $h1_xml+1 ;$p < @zeilen_xml ; $p++){
			if($zeilen_xml[$p] =~ /\<\/prqaproject\>/){
				last;	# verlasse innere for-Schleife
			}
		$h2_xml++
		}
		last;	# verlasse aeussere for-Schleife
	}	
	$h1_xml++;
}	

print $h1_xml."\n";
print $h2_xml."\n";
	
open(SCHREIBEN,"> prqaproject.xml")
or die "Fehler beim Oeffnen von 'prqaproject.xml': $!\n";


foreach(@zeilen_xml){
	
	if($z1 < $h1_xml+1){	# schreibe die Anfangszeilen zu der Datei
		print SCHREIBEN $zeilen_xml[$z1];
	} else {
		foreach(@var_arr_files){	# schreibe die Namen der Files in die Zieldatei
			chomp($_);
			print SCHREIBEN "$string1".$_."$string2\n";
		}
		print SCHREIBEN " \<\/files\>\n\<\/prqaproject\>";		
		last;
	}
	$z1++;	
}

close(SCHREIBEN) or die "Fehler beim Schliessen von 'prqaproject.xml': $! \n";
\end{lstlisting}
%
\chapter{Pr�fstandsmessdaten}\label{cha:PruefMess}
%
In folgender Tabelle sind sowohl die Luftpfadsgr��en des Einlassbeh�lters als auch der Winkel der Haupteinpritzung aufgelistet, die im Betriebspunkt ($n_{mot}=1500min^{-1}$; $q_{inj}=10mm^3/inj$) variiert und vermessen wurden.
%
\begin{longtable}{|l|l|l|l|l|l|l|}
\hline
Messung Nr. & $s\phi_{HE}$[�KW v.OT] & $\lambda$ & $T_{2EB}$[K] & $\frac{dm_L}{dt}[\frac{kg}{s}]$ & $\frac{dm_{Gas}}{dt}[\frac{kg}{s}]$ & $x_{AGR}$ [-]\\
\endhead
\multicolumn{7}{c}{}\\
\hline
Messung Nr. & $s\phi_{HE}$[�KW v.OT] & $\lambda$ & $T_{2EB}$[K] & $\frac{dm_L}{dt}[\frac{kg}{s}]$ & $\frac{dm_{Gas}}{dt}[\frac{kg}{s}]$ & $x_{AGR}$ [-]\\
\hline
\endfirsthead
\multicolumn{7}{c}{}
\endfoot
\multicolumn{7}{c}{}
\endlastfoot
1 & -1 & 4.8 & 291 & 0.024 & 0.024 & 0 \\
\hline
2 & 1 & 3.7  & 291.4 & 0.024 & 0.024 & 0 \\
\hline
3 & 3 & 3.6  & 291.4 & 0.024 & 0.024 & 0 \\
\hline
4 & 5 & 3.58  & 291.6 & 0.024 & 0.024 & 0 \\
\hline
5 & 7 & 3.55  & 291.3 & 0.024 & 0.024 & 0 \\
\hline
6 & 9 & 3.52 & 291.2 & 0.024 & 0.024 & 0 \\
\hline
7 & 11 & 3.56  & 291.5 & 0.024 & 0.024 & 0 \\
\hline
 &  &  &  &  &  &  \\
\hline
8 & 1 & 3.35  & 291.5 & 0.02 & 0.025 & 0.2  \\
\hline
9 & 3 & 3.1  & 292.3 & 0.02 & 0.025 & 0.2  \\
\hline
10 & 5 & 3.03  & 292.5 & 0.02& 0.025 & 0.2  \\
\hline
11 & 7 & 3   & 292.2 &  0.02&0.025& 0.2  \\
\hline
12 & 9 & 3   & 291.8 &  0.02& 0.025 & 0.2  \\
\hline
13 & 11 & 3   & 291.9 &  0.02& 0.025 & 0.2  \\
\hline
 &  &  &  &  &  &  \\
\hline
14 & 1 & 2.3  & 338 & 0.013 & 0.022 & 0.041  \\
\hline
15 & 3 & 2.09  & 340.2 & 0.013 &0.022 & 0.041  \\
\hline
16 & 5 & 2.06  & 340.8 & 0.013 & 0.022 & 0.041  \\
\hline
17 & 7 & 2.03  & 330 & 0.013 & 0.022 & 0.041  \\
\hline
18 & 9 & 1.98  & 338 & 0.013 & 0.022 & 0.041  \\
\hline
19 & 11 & 2.02  & 337 & 0.013 & 0.022 & 0.041 \\
\hline
20 & 13 & 2.04  & 336 & 0.013 & 0.022 & 0.041  \\
\hline
21 & 15 & 2.06  & 335.5 & 0.013 & 0.022 & 0.041  \\
\hline
 &  &  &  &  &  &  \\
\hline
22 & 5 & 1.92  & 3.5 & 0.01 & 0.023 & 0.565  \\
\hline
23 & 7 & 1.71  & 315.5 & 0.01 & 0.023 & 0.565   \\
\hline
24 & 9 & 1.62  & 315 & 0.01 & 0.023 & 0.565   \\
\hline
25 & 11 & 1.62  & 315 & 0.01 & 0.023 & 0.565   \\
\hline
26 & 13 & 1.64  & 313 &0.01& 0.023 & 0.565   \\
\hline
27 & 15 & 1.65  & 313 & 0.01& 0.023 & 0.565   \\
\hline
28 & 17 & 1.68  & 312.5 & 0.01& 0.023 & 0.565   \\
\hline
29 & 19 & 1.68  & 313 & 0.01 & 0.023 & 0.565   \\
\hline
30 & 21 & 1.7  & 313 & 0.01 & 0.023 & 0.565   \\
\hline
 &  &  &  &  &  &  \\
\hline
31 & 19 & 1.33  & 338 & 0.008 & 0.021 & 0.62  \\
\hline
32 & 21& 1.34  & 338 & 0.008 & 0.021 & 0.62  \\
\hline
33 & 23& 1.36  & 338 & 0.008 & 0.021 & 0.62  \\
\hline
34 & 25& 1.36  & 338 & 0.008 & 0.021 & 0.62  \\
\hline
35 & 27& 1.36  & 336 & 0.008 & 0.021 & 0.62  \\
\hline
36 & 29 & 1.37  & 337 & 0.008 & 0.021 & 0.62  \\
\hline
\caption{Luftpfadsgr��en des Einlassbeh�lters und Winkel der Haupteinpritzung, bei denen die Druckverl�ufe am Pr�fstand gemessen wurden. Betriebspunkt ($n_{mot}=1500min^{-1}$; $q_{inj}=10mm^3/inj$).}
\label{tab:TabPruef}
\end{longtable}

